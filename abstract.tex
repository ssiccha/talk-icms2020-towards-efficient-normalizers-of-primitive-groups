Guten Tag Frau Prof. Eick,

wie wir in Cambridge besprochen hatten, möchte ich mich gerne für die
ICMS Session "Groups and group actions" anmelden und einen Vortrag
halten. Aller Voraussicht nach werde ich im Juli an der TU
Kaiserslautern angestellt sein.

Title:
Normalizers of primitive groups with non-regular socle in polynomial time

Abstract:
The \emph{normalizer problem} has as input generating sets $X$ and
$Y$ for subgroups $G$ and $H$ of the symmetric group $S_n$, and asks
one to return generators for $N_H(G)$.

Polynomial time solutions are known for many permutation group
problems. However, in addition to the normaliser problem, polynomial
time algorithms for computing set stabilisers, centralisers and
intersections of permutation groups have so far proven elusive. The
latter three problems are polynomial time equivalent and their
equivalence class is called the \emph{Luks class}. While solving these
problems is of direct practical interest, they also yield interesting
phenomena from a complexity theoretical point of view. The famous
graph isomorphism problem reduces in polynomial time to the problems
in the Luks class, which in turn reduce to the normalizer problem.
However, no reductions in the reverse direction are known.

For a primitive group with non-regular socle $G$ we present an
algorithm to compute the normalizer $N_{S_n}(G)$ both practically
efficiently and in polynomial time. For primitive groups $G$ of the
above type, this yields a reduction of the normalizer problem to the
Luks class.

Freundliche Grüße,
Sergio Siccha
