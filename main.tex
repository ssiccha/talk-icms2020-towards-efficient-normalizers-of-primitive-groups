 \documentclass{beamer}
%\documentclass[handout]{beamer}
\usetheme{metropolis}

%\setbeameroption{show notes}
%\setbeameroption{show notes on second screen}

% Packages
\usepackage[utf8]{inputenc}

\usepackage{animate}

\usepackage{appendixnumberbeamer}
\usepackage{xcolor}
\usepackage{multirow}
% Maths
\usepackage{amsthm}
\usepackage{tikz}
\usetikzlibrary{cd}

% References
\usepackage{biblatex}
\addbibresource{./references.bib}

% Package Settings
% AMSTHM
\theoremstyle{plain}
\newtheorem{thm}{Theorem}[section]
\newtheorem{mylemma}[thm]{Lemma}
\newtheorem{cor}[thm]{Corollary}
\newtheorem{rem}[thm]{Remark}
\newtheorem{alg}[thm]{Algorithm}

\theoremstyle{definition}
\newtheorem{defn}[thm]{Definition}

\newcommand{\todo}[1]{{\color{red}*** #1 ***}}

\renewcommand{\phi}{\varphi}

% BEAMER
% Hack for only and tabular
\newcommand{\tabularonly}{AS}
% Hack for tilde and hat
\newcommand{\mytilde}[1]    {\ensuremath{\tilde{#1}}}
\newcommand{\myhat}[1]    {\ensuremath{\widehat{#1}}}
% Hack to display \Gamma with a little bit of space to the left
% Otherwise \ind _ \Gamma ^ H looks really weird
\let\oldGamma\Gamma
\renewcommand{\Gamma}{{\hspace{0.5pt} \oldGamma}}
% Hack to reference the notation-definition in "lift a.s. normaliser"
\newcommand{\refdef}    {{(6.1)}}

% Symbols
% Write function definitions like
% f \from A \to B
\newcommand{\from}	{\ensuremath{\colon}}
\renewcommand{\to}	{\ensuremath{\rightarrow}}
\newcommand{\into}	{\ensuremath{\hookrightarrow}}
\newcommand{\onto}	{\ensuremath{\twoheadrightarrow}}
\newcommand{\iso}	{\ensuremath{\xrightarrow{\,\raisebox{-1pt}{\ensuremath{\scriptstyle{\sim}}}\,}}}
\newcommand{\N}	{\ensuremath{\mathbb N}}
\newcommand{\Z}	{\ensuremath{\mathbb Z}}
\newcommand{\Q}	{\ensuremath{\mathbb Q}}
\newcommand{\Sn}{{\ensuremath{\mathrm{S}_n}}}
\newcommand{\uln} {{\ensuremath{\underline n}}}

%%% Categories
\newcommand{\permgrp}   {\ensuremath{\mathbf{PermGrp}}}

%%% Unary Operators
% Functions
\DeclareMathOperator{\dom}{dom}
\DeclareMathOperator{\id}{id}
\DeclareMathOperator{\im}{Im}
% Groups
\DeclareMathOperator{\sym}{Sym}
\DeclareMathOperator{\Sym}{Sym}
\DeclareMathOperator{\alternating}{Alt}
\DeclareMathOperator{\Alt}{Alt}
\DeclareMathOperator{\soc}{soc}
\DeclareMathOperator{\aut}{Aut}
\DeclareMathOperator{\Aut}{Aut}
\DeclareMathOperator{\diag}{diag}
\DeclareMathOperator{\out}{Out}
\DeclareMathOperator{\Out}{Out}
\DeclareMathOperator{\ind}{Ind}
\DeclareMathOperator{\res}{Res}
\DeclareMathOperator{\stab}{Stab}
\newcommand{\hatOperator}   {\ensuremath{\, \myhat \cdot\,}}
\def\Norm#1#2{\mathrm{N}_{#1}(#2)}

%%% Binary Operators %%%
\newcommand{\union}{\mathbin{\cup}}
\newcommand{\bigunion}{\mathbin{\bigcup}}
\newcommand{\disjointunion}{\mathbin{\uplus}}
\newcommand{\intersection}{\mathbin{\cap}}
\newcommand{\bigintersection}{\mathbin{\bigcap}}
\newcommand{\norm}{\mathbin{\lhd}}
\newcommand{\characteristic}{\mathbin{\text{char}}}


% New commands
\newcommand{\abs}[1]	{%
	\ensuremath{
		\left| #1 \right|
	}%
}
\newcommand{\gen}[1]	{
	\ensuremath{
		\left\langle \, #1 \, \right\rangle
	}
}
\newcommand{\set}[1]	{
	\ensuremath{
		\left\{ #1 \right\}
	}
}
\newcommand{\sset}[2] {
	\ensuremath{
		\left\{ \left.\,
			#1
		~\right|~
			#2
		\,\right\}
	}
}


\title{Towards Efficient Normalizers For Primitive Groups}
\date{July 15th, 2020}
\author{Sergio Siccha}
\institute{TU Kaiserslautern}

\begin{document}

\maketitle
\note{
VERSION WITH NOTES!
}

\section{Introduction}
%\begin{frame}{Theory and Practice}
%\TODO{eigentlich will ich nur sagen, dass ich sowohl theoretisch als auch
%praktisch effizient bin. zwei versionen von algorithmen. wenn ich irgendwo
%einen theo alg benutze, dann sage ich das}
%Let $G = \gen X \leq \Sym \Omega$, and $n := \abs \Omega$.
%We consider
%\begin{itemize}
%\item deterministic polynomial time $O(\abs X n ^ c)$
%\item Las Vegas nearly-linear time $O(\abs X n \, (\log n) ^ c)$
%\end{itemize}
%\vspace{1em}
%
%\pause
%We can compute in polynomial and in nearly-linear time:
%\begin{itemize}
%\item the order $\abs G$
%\item a composition series of $G$
%\item the socle of $G$
%\item \ldots
%\end{itemize}
%\end{frame}

\begin{frame}{Normalizers}
Let $G, M \leq K$ be groups.
The \emph{normalizer of $G$ in $M$} is
the group
\[
N_M(G) = \set{ m \in M ~|~ G ^ m = G }.
\]

\pause
In practice partition backtrack.

Best known theoretical algorithm by D. Wiebking for
$N_{\Sym \Omega}(G)$ in
time $2 ^ {O(n)}$.
\end{frame}

\note{
theoretical,
canonical labelling
}

\begin{frame}{Related Work}
\todo{what other polynomial time results are there?}
\end{frame}

\begin{frame}{Primitive Groups}
A transitive group $G \leq \Sym \Omega$ is
\emph{imprimitive}, if there exists a non-trivial $G$-invariant partition of
$\Omega$,
otherwise it is \emph{primitive}.

\begin{columns}
\begin{column}{0.5\textwidth}
$D_8 := \gen{ (1,2,3,4),\, (1,4)(2,3) }$
\\[1em]

$\set{ \set{1,3}, \set{2,4} }$ is $D_8$-invariant.
\end{column}
\begin{column}{0.3\textwidth}
{
\only<1->{
\begin{figure}[h]
\centering
\includegraphics[width=0.9\textwidth]{./pictures/square.png}
\caption{A square}
\end{figure}
}
}
\end{column}
\end{columns}
\end{frame}

\begin{frame}%{Resultat}
\begin{block}{Theorem (S. 2020)}
Let $G = \gen X \leq \Sym \Omega$ be a primitive group
with non-regular socle.
We can compute (a generating set of) $N_{\Sym \Omega}(G)$
in time $O(\abs X \abs \Omega ^ c)$.
\end{block}
\vspace{1em}
Note: Most parts in nearly-linear time.

\vspace{1em}
\pause
Note: For $M \leq \sym \Omega$ we have
$N_{M}(G) = M \cap N_{\sym \Omega}(G)$.
\end{frame}

\begin{frame}{Implementation}
A partial implementation is available at the github repository
\begin{center}
\href{https://github.com/ssiccha/NormalizersOfPrimitiveGroups}
{\textcolor{blue}{ssiccha/NormalizersOfPrimitiveGroups}}
\end{center}
\vspace{1\baselineskip}

$N_{\Sym \Omega}(G)$ for $G$ primitive with $\abs \Omega = 5 ^ 7 = 78125$ in
$\approx 67 s$.
\end{frame}


\section{Strategy}
\begin{frame}{Standing on the shoulders of giants}
The algorithm depends heavily on
\begin{itemize}
\item the O'Nan-Scott Theorem
\item the classification of finite simple groups (CFSG)
\end{itemize}
\end{frame}

\begin{frame}{Permutation isomorphism}

\todo{diagram version}

Let $G \leq \sym \Omega$ and $H \leq \sym \Delta$.
We call $(f, \varphi)$ with
isomorphisms
$f \from \Omega \iso \Delta$
and
$\varphi \from G \iso H$
a \emph{permutation isomorphism}
from $G$ to $H$
if
$\forall g \in G, ~ \forall \omega \in \Omega :$
\[
    f(\omega ^ g) = f(\omega) ^ {\varphi(g)}.
\]
\end{frame}


\begin{frame}{Classification of primitive groups}
Let $G$ be a group. The \emph{socle} of $G$, denoted $\soc G$,
is the group generated by all minimal normal subgroups of $G$.

The socle of a primitive group is characteristically simple.
\pause
\begin{block}{O'Nan-Scott Theorem}
    Let $G \leq \sym \Omega$ be primitive.
    We know all possible permutational isomorphism types of
    \vspace{-0.5em}
    \begin{itemize}
        \item
        $\soc G$,
        \item
        $N_{\sym \Omega}(\soc G)$.
    \end{itemize}
\end{block}
\end{frame}

\note{
Typ PA: einfachster nicht-trivialer Fall.
}

\begin{frame}{Main steps}
Let $G \leq \sym \Omega$ be primitive.
Compute $N_{\sym \Omega}(G)$.

Compute:
\begin{itemize}
\item Step 1: $M := N_{\sym \Omega}(\soc G)$.
Then we have $N_{\sym \Omega}(G) = N_M(G)$.

\item Step 2: Grouphom.
$
    \rho : M \to S_k
$
with $k \ll \abs \Omega$.
\end{itemize}
\vspace{1em}

\pause
Today: focus on step 1.
\end{frame}

\begin{frame}{Wreath products}
\begin{defn}
    Let $H$ be a group and let $K \leq S_\ell$.
    $K$ acts on $H ^ \ell$ by permuting components.
    The group
    $H \wr K := H ^ \ell \rtimes K$
    is the \emph{wreath product of $H$ with $K$}.
\end{defn}

\pause
If $H \leq \Sym \Delta$, then $H \wr K$ acts faithfully on
\begin{itemize}
\item $\Delta \times \set{1, \ldots, \ell}$ via the \emph{imprimitive action}
\item $\Delta ^ \ell$ via the \emph{product action}
\end{itemize}

$i$-th copy of $H$ acts on $i$-th copy of $\Delta$, $K$ permutes the copies.
\end{frame}

\note{
    Make sure to explain intuition behind wreath products
}

%\begin{frame}{AS Type}
%Let $G \leq \sym \Omega$ be primitive.
%We say $G$ is of \emph{type AS}, if
%there exist a non-regular non-abelian simple group
%$T \leq \sym \Omega$ with $\soc G = T$.
%
%Use constructive recognition to compute $N_{\sym \Delta}(T)$.
%\end{frame}

\begin{frame}{PA Type}
Let $G \leq \sym \Omega$ be primitive.
We say $G$ is of \emph{type PA}, if
there exist a non-abelian simple group
$T \leq \sym \Delta$
and a permutation iso.
$(f : \Omega \to \Delta ^ \ell,~ \phi : G \to H)$
with
\[
\phi (\soc G) = T ^ \ell \leq \sym(\Delta ^ \ell).
\]

\pause
\begin{block}{Corollary}
$G \leq \sym(\Delta ^ \ell)$ primitive s.t.
$\soc G = T ^ \ell$.
Let $M := N_{\sym (\Delta ^ \ell)}(T ^ \ell)$.
Then
\[
    M = N_{\sym \Delta}(T) \wr S_\ell
\]
and thus $M / (\soc G) \into \Out(T) \wr S_\ell$.
\end{block}
\end{frame}

\begin{frame}{PA Type}
\begin{block}{Step 1}
Let $G \leq \sym(\Delta ^ \ell)$ primitive with
$\soc G = T ^ \ell$.
Then $M = N_{\Sym \Delta}(T) \wr S_\ell$ can be computed in polynomial
time.
\end{block}
\vspace{1em}

\pause
\begin{block}{Step 2}
\vspace{-1em}
\[
    M \to \Sym( \Delta \times \set{1, \ldots, \ell} ).
\]
\pause
Theoretical version: Let $d := \abs \Delta$.
Then $\abs{\Out T} \leq 3 \log d$ by
\cite{guralnick-maroti-pyber:normalizers-primitive-groups}
and $\ell = \log_d(n)$.
\[
    M \to \Out T \wr S_\ell \to S_{(3 \log n)}.
\]
\end{block}
\end{frame}

\begin{frame}
How to find $T ^ \ell$ if $G \leq \Sym \Omega$ is arbitrary?
Find $\ell$ \emph{projections} $\pi_i : \soc G \onto T$
such that
\[
\prod \pi_i : \soc G \iso T ^ \ell
\]
yields a permutation isomorphism.
\end{frame}

\section{Permutation morphisms}
\begin{frame}{Permutation morphisms}
Let $G \leq \sym \Omega$ and $H \leq \sym \Delta$.
We call $F = (f, \varphi)$ with
$f \from \Omega \to \Delta$
and
$\varphi \from G \to H$
a \emph{permutation morphism}
if
$\forall g \in G ~ \forall \alpha \in \Omega :$
\[
    f(\alpha ^ g) = f(\alpha) ^ {\varphi(g)}.
\]

\pause
\begin{block}{Theorem (S.)}
Permutation groups together with permutation morphisms and their component-wise
composition form a category.
That category admits finite products.
\end{block}
\end{frame}

\begin{frame}{Permutation morphisms}
\begin{block}{Lemma (S.)}
Let $G \leq \sym \Omega$ and $f : \Omega \onto \Delta$.
There exist a (unique) group $H \leq \Sym \Delta$ and a (unique)
group epim. $\phi : G \to H$ such that
$(f, \phi)$ is a permutation epimorphism
iff.
\[
    \set{ f ^ {-1}(\set{\delta}) ~|~ \delta \in \Delta }
\]
is $G$-invariant.
\end{block}

\pause
If $G$ transitive: think the action induced on the above block system.
\end{frame}

%\begin{frame}{Primitive groups and permutation morphisms}
%Note:
%a group $G$ is simple iff. all group homomorphisms
%$\phi : G \to H$ are trivial or injective.
%
%\begin{block}{Lemma (S.)}
%A permutation group $G$ is primitive iff. all permutation morphisms
%$F : G \to H$ are trivial or injective. we have $\ker \phi = \set{1_G}$ or $\ker \phi = G$.
%\end{block}
%a group admits only trivial group homomorphisms and group isomorphisms
%\end{frame}

\begin{frame}{Permutation morphisms and products}
\begin{block}{Example}
$\Omega := \set{1, \ldots, 4}$ und
$V := \langle a := (1,2)(3,4),~$%
$b := (1,3)(2,4) \rangle$.

\begin{figure}[H]
\centering
\begin{tikzpicture}
    % nodes
    \foreach \x in {1, ..., 2}
        \foreach \y in {1, ..., 2}
            \draw [fill] (\x, \y) circle [radius=0.050];
    \node [above left] at (1, 2) {{$1$}};
    \node [above left] at (2, 2) {{$2$}};
    \node [above left] at (1, 1) {{$3$}};
    \node [above left] at (2, 1) {{$4$}};
\end{tikzpicture}
\end{figure}

\only<1>{\vspace{9.25em}}
\only<2>{
\begin{align*}
p_1 &: \Omega \to \set{1,3},~~
1,2 \mapsto 1, ~~ 3,4 \mapsto 3,
\\
\pi_1 &: G \to \gen{(1,3)},~~
\id_\Omega, a \mapsto \id_{\set{1,3}},
~~b, ab \mapsto (1,3).
\end{align*}

Analoguously $P_2$.
Then $P_1 \times P_2 : V \iso \gen{(1,3)} \times \gen{(1,2)}$
}
\only<3>{
For
$p_1' : \omega \mapsto \omega ^ {\gen{a}}$
we have
\begin{align*}
1 ^ {\gen a} = 2 ^ {\gen a} = \set{ 1,2 }
\\
3 ^ {\gen a} = 4 ^ {\gen a} = \set{ 3,4 }
\end{align*}
\vspace{0.75em}
}
\end{block}
\end{frame}

%\begin{frame}{Permutation morphisms and products}
%$\Delta = \set{1, \ldots, 5}$, $\Omega = \Delta ^ 2$,
%$H = (A_5) ^ 2 \leq \Sym(\Delta ^ 2)$,
%$N_1 := \mathbf 1 \times A_5 \unlhd H$.
%\end{frame}

\begin{frame}[standout]
Thanks!
\end{frame}

\printbibliography
\appendix
\end{document}
